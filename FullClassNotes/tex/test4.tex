

\marginpar{Date: 04.13.15 } % Continue notes 
Here we address the possibility of the char poly \( \rho(x) \) of \(
\sum_{k=0}^{n} a_ky^{(k)} = 0  \) with complex zeros. So, assuming that
\( r = a + bi \) is a zero of \( \rho = \sum_{k=0}^{n} a_k x^k \), then
we know that \( y= e^{rk} = e^{(a+bi)x} = e^{ax} e^{(bx)i}\). Recall
that if \( a_k \in \mathbb{R} \) for \( 0 \leq k \leq n \), i.e., \(
\rho(x) \in \mathbb{R}[x] \), then \( \bar{r} = a-bi\) \( (b \pm 0 \) is
also a zero of \( \rho(x) \). Thus, in the case of \( r = a \pm bi \),
we are "missing" now not just one real-valued basis element, but two
real valued basis elements. 

Euler's identity enabels us to determine there two missing basis
elements: 

\[ \text{ (1) } e^{ix} = \cos(x) = i\sin(x) \in \mathbb{C} \text{ for all } x \in
\mathbb{R} \]

what led Euler to this identity are the Maclarin series: 

\[ e^x = 1 + \frac{x}{1!} + \frac{x^2}{2!} + \frac{x^3}{3!}+
\frac{x^4}{4!}+ \frac{x^5}{5!} + \cdots = \sum_{n \geq 0} \frac{x^n}{n!};\]

\[ \cos(x) = 1 - \frac{x^2}{2!} +
\frac{x^4}{4!} - \frac{x^6}{6!} + \frac{x^8}{8!} - \frac{x^10}{10!} +
\cdots = \sum_{n \geq 0} (-1)^n \frac{x^{2n}}{2n!};\]

\[ \sin(x) =  x -  \frac{x^3}{3!} + \frac{x^5}{5!} - \frac{x^7}{7!} +
\frac{x^9}{9!} - \frac{x^11}{11!} + \cdots = \sum_{n \geq 0} (-1)^n
\frac{x^{2n +1}}{(2n+1)!};\]

These hold for all \( x \in \mathbb{R} \). Notice that (1) simply gives
pts on the unit circle in \( \mathbb{C} \)

\[ |e^{ix}| = 1 \text{ for all } x \in \mathbb{R} \]

Note: \( e^{i \pi} + 1 = 0. \) \( (x = \pi) \)

Observe that \( e^{ix} \) is a complex-valued scalar mult of \( \cos x
\) \& \( \sin x \). So, back to the soln

\begin{align*}
  y &= e^{ax} e^{(bx)i} \\
  &=e^{ax}(\cos bx + i \sin bx) \\
  &= e^{ax} \cos x + i(e^{ax} \sin bx),
\end{align*}

which is likewise a complex-valued linear combo of \( e^{ax} \cos bx \) \& \(
e^{ax} \sin bx \). These are to real-balued soln to the ODE \(
\sum_{k=0}^{n}a_ky^{(k)} =0, \) i.e., there are the two missing basis
elements. therefore if \( (x-r)^m, \) where \( r = a +bi \), is a factor
of \( \rho(x) \), then the corespoinding basis elements are: 

\begin{align*}
  e^{ax} \cos bx &\text{ , } e^{ax} \sin bx \\
  xe^{ax} \cos bx &\text{ , } xe^{ax} \sin bx \\
  x^2e^{ax} \cos bx &\text{ , } x^2e^{ax} \sin bx \\
  \vdots & \vdots  \\
  x^{m-1}e^{ax} \cos bx &\text{ , } x^{m-1}e^{ax} \sin bx
\end{align*}

Here there are the missing 2m many basis elements. 

\marginpar{Date: 04.14.15 } % Continue notes 

\begin{example}
  \[ y'' + y' - y = \sin^2x \]
  Here \( \sum_{k=0}^{n} y^{(k)} = f(x) \), where \( n=2 \) and \( f(x) =
  \sin^2x\). Notice that 

  \begin{align*}
    y &= \sin^2x \implies \\
    y' &= 2 \sin x \cos x \implies \\
    y'' &= 2 \cos^2x - 2\sin^2x.
  \end{align*}

  \begin{align*}
    y &= \sin^2x; \\
    y' &= 2 \sin x \cos x = \sin 2x \\
    y'' &= 2 \cos^2x - 2\sin^2x = 2 \cos 2x.
  \end{align*}

  the terms in there derivatives are 

  \begin{align*}
    1, \\
    \sin^2x, \\
    \sin x \cos x \\
    \cos^2x. 
  \end{align*}

  Another way to see these terms is as 

  \begin{align*}
    1, \\
    \cos 2x \\
    \sin 2x. 
  \end{align*}

  We consider a posible particular soln 

  \[ y_p = a + b\cos 2x + c \sin 2x, \]

  which is a linear combo of the terms above. Note that \( y_p \) is a
  particular soln iff 

  \[ y_p'' + y_p' - y_p = \sin^2x \]

  Now, 

  \begin{align*}
    y_p &= a + b\cos 2x + c \sin 2x \implies \\
    y_p' &= -2b\sin2x + 2c\cos2x \implies \\
    y_p'' &= -4b\cos2x -4c\sin2x \implies
  \end{align*}

  \[ y_p'' + y_p' - y_p = \sin^2x \iff \]

  \[ (2c-4b)\cos2x - (2b+4c)\sin2x - b\cos2x -c\sin2x -a = \sin^2x \iff \]
  \[ (2c-5b)\cos2x - (2b+5c)\sin2x -a = \frac{1-\cos2x}{2} \iff \]
  \[ -a-1/2+(2c-5b+1/2)\cos2x - (2b+5c)\sin2x = 0. \]

  This last equation is an identity if 

  \[ a = -1/2 \text{ \& } 
  \begin{cases}
    -5b + 2c + 1/2 &= 0 \\
    2b + 5c &=0.
  \end{cases}
  \]

  Note:
  \[
  \begin{cases}
    -10b + 4c &= -1 \\
    2b + 5c &= 0
  \end{cases}
  \]
\[
  \begin{bmatrix}
    10 & 25 & | & 0 \\
    -10 & 4 & | & -1 \\
  \end{bmatrix}
\]

  \begin{align*}
    c &= -1/29 \\
    b &= 5/58
  \end{align*}

  Therefore

  \[ y_p = -1/2 + \frac{5}{58} \cos2x - \frac{1}{29}\sin2x, \]

  Which you have cheked is an actual soln to \( y'' + y' -y = \sin^2x. \)

\end{example}

\marginpar{Date: 04.15.15 } % Continue notes 

\begin{example}
  \[ y'' + y' - y = \sec x \]

  \begin{align*}
    f(x) &= \sec x \\
    f'(x) &= \sec x \tan x \\
    f''(x) &= \sec x \tan^2x + \sec^3x \\
    &= \sec x (\sec^2x - 1) + \sec^3x \\
    &= 2\sec^3x - \sec x
  \end{align*}

  Put 

  \[ B = \{ 1, \sec x, \sec x \tan x, \sec^3x \}  \]

  Also, put 

  \[ y_p = a + b \sec x + c\sec x \tan x + d \sec^3x \]

  \[ y_p'  \]

  Since \( \sec^3x \tan x \notin  \) Span \( B \), the method of
  undetermined coefis fails. \\[5mm]

  Idea: Consider \( B = \{ y_1, \dots, y_n \} \), the basis of \(
  \sum_{k=0}^{n} a_k y^{(k)} = 0 \), and 

  \[ \text{ (1) } y_p = \sum_{k=1}^{n} v_k(x) y_k(x). \]

  We will show that a particular soln of \( \sum_{k=0}^{n} a_k y^{(k)}
  = f(x)  \) always has form (1) for some \( v_k(x) \in C^{(n)} (I) \).
  this is called " Variation of parameters."

\end{example}

\marginpar{Date: 04.16.15 } % Continue notes 

We want a particular soln \( y = v_1(x) y_1(x) + v_2(x) y_2(x) \) to \(
\sum_{k=1}^{2} a_k y^{(k)} = f(x) \), where the hom soln space has basis
\( B = \{ y_1, y_2 \}  \subset C^{(2)}(I) \). Here we require 

\[  y_p'' + py_p' + qy_p = f(x) \text{ (monic). } \]

\[ y_p = v_1y_1 + v_2y_2 \implies \]

\[ y_p' = v_1'y_1 + v_1y_1' + v_2'y_2 + v_2y_2' \]

If \( v_1'y_1 + v_2'y_2 = 0 \) then 

\[ y_p' = v_1y_1' + v_2y_2' \]

Thus, 

\[ y_p'' = v_1'y_1' + v_1y_1'' + v_2'y_2' v_2y_2'' \]

\[ y_p'' +py_p' + qy_p = f(x) \iff \]

\[ v_1'y_1' + v_1y_1'' + v_2'y_2' v_2y_2'' + pv_1y_1' + pv_2y_2' +
qv_2y_1 + qv_2y_2 = f(x) \iff \]

\[ v_1(y_1'' + py_1' + qy_1) + v_2(y_2'' + py_2' + qy_2) + v_1' y_1' +
v_2'y_2' = f(x) \iff \]

\[ v_1'y_1' + v_2'y_2' = f(x). \]

therefore

\begin{cases}
  y_1' v_1' + y_1'v_2' = f(x) \\
  y_1v_1' + y_2v_2' = 0
\end{cases}

iff 

\[ 
\begin{pmatrix}
  y_1 & y_2 \\
  y_1'& y_2' 
\end{pmatrix}
\begin{pmatrix}
  v_1'\\
  v_2' 
\end{pmatrix}
=
\begin{pmatrix}
  a\\
  b
\end{pmatrix}
\]

% missed some stuff

Therefore 

\[ v_1 = - \int \frac{y_2(x)f(x)}{|w(x)|} dx \]

and 

\[  v_2 =  \int \frac{y_1(x)f(x)}{|w(x)|} dx \]

Hence, 

\[ y_p(x) =  \]

% missed some more stuff

Now, in general, consider 

\[ \sum_{k=0}^{n} a_ky^{(k)} = f(x) \]
with basis \( B = \{ y_j \in C^{(n)} (I) | 1 \leq j \leq n \} \) for the
assoc hom ODE. Also, consider 

\[ y_p= \sum_{j=1}^{n} v_jy_j, \]
then
\begin{align*}
  y_p' &= \sum_{j=1}^{n}(v_j'y_j +v_jy_j') \\
  &= \sum_{j=1}^{n} v_jy_j' \text{ , if } \sum_{j=1}^{n}v_j'y_j = 0.
\end{align*}

Now, 

\begin{align*}
  y_p'' &= \sum_{j=1}^{n}(v_j'y_j' + v_jy_j'') \\
  &= \sum_{j=1}^{n}y_jy_j'' \text{ , if } \sum_{j=1}^{n}v_j'y_j' =0
\end{align*}

continuing, 

\[ y_p^{(k)} = \sum_{j=1}^{n} v_jy_j^{(k)} \text{ , if }
\sum_{j=1}^{n}v_j'y_j^{(k-1)} =0 \]

for \( 1 \leq k \leq n-1 \). Finally, 

\begin{align*}
  y_p^{(n)} &= \sum_{j=1}^{n} (v_j'y_j^{(n-1)} +v_jy_j^{(n)}) \\
 &=    \sum_{j=1}^{n} v_j'y_j^{(n-1)} + \sum_{j=1}^{n}v_jy_j^{(n)}) 
\end{align*}

Therefore

\[ \sum_{k=1}^{n}a_ky_p^{(k)} = f(x) \iff \]

\[ a_n \sum_{j=1}^{n} v_j'y_j^{(n-1)} + a_n  \sum_{j=1}^{n}v_jy_j^{(n)}
+ \sum_{k=1}^{n-1}a_k \sum_{j=1}^{n}v_jy_j^{(k)} = f(x) \iff\]

\[  a_n \sum_{j=1}^{n} v_j'y_j^{(n-1)} +  \sum_{j=1}^{n} a_nv_jy_j^{(n)}
+ \sum_{j=1}^{n} \sum_{k=1}^{n=1} a_kv_jy_j^{(k)} = f(x) \iff
 \]

 \[ 
  a_n \sum_{j=1}^{n} v_j'y_j^{(n-1)}  + \sum_{j=1}^{n}v_j \sum_{k=1}^{n}
  a_ky_j^{(k)} = f(x) \iff
 \]

 \[ a_n \sum_{j=1}^{n} v_n'y_j^{(n-1)} = f(x) \]

 since \( y_j \in B. \) Thus, 

 \begin{align*}
   \sum_{j=1}^{n}v_j'y_j &= 0  \text{ 1st eqn }\\
   \sum_{j=1}^{n}v_j'y_j' &= 0 \text{ 2nd eqn }\\
   \vdots \\
   \sum_{j=1}^{n}v_j'y_j^{(k-1)} &= 0 \text{ kth eqn }\\
   \vdots \\
   \sum_{j=1}^{n}v_j'y_j^{(n-2)} &= 0 \text{ (n-1)th eqn }\\
   a_n\sum_{j=1}^{n}v_j'y_j^{(n-1)} &= f(x) \text{ nth eqn }\\
 \end{align*}





Now, 
\begin{align*}
  y_p' &= v_1'y_1 + v_1y_1' + v_2'y_2 + v_2y_2' \implies \\
  y_p'' &= v_1''y_1 + v_1'y_1' + v_1'y_1' + v_1y_1'' + 
v_2''y_2 + v_2'y_2' + v_2'y_2' + v_2y_2'' \\
&= 2(v_1'y_1' + v_2'y_2') + v_1y_1'' + v_2y_2'' ,  
\end{align*}

if we impose the condition \( v_1''y_1 + v_2''y_2 = 0. \) therefore 

\begin{align*}
  f(x) &= y_p'' + py_p' + qy_p \\
  &= 2(v_1'y_1' + v_2'y_2') + v_1y_1'' + v_2y_2'' +
  pv_1'y_1 + pv_1y_1' + pv_2'y_1 _ pv_2y_2' + q_vy_1 + qv_2y_2\\
  &= v_1(y_1'' + py_1' +qy_1) + v_2(y_2'' + py_2' +qy_2) + 2(v_1'y_1' +
  v_2'y_2') + p(v_1'y_1'+v_2'y_2) \\
  &= 2(v_1'y_1' + v_2'y_2') + p(v_1'y_1 + v_2'y_2)
\end{align*}

therefore, 

\[ f(x) =  2(v_1'y_1' + v_2'y_2') + p(v_1'y_1 + v_2'y_2)
\]

implies 

\[ f(x) = v_1'(2y_1' +py_1) + v_2'(2y_2' +py_2) \]


% Date: 04.20.15 ------------------ {{{
\newpage
\marginpar{Date: 04.20.15 }

We want a particular soln \( y = v_1(x) y_1(x) + v_2(x) y_2(x) \) to \(
\sum_{k=1}^{2} a_k y^{(k)} = f(x) \), where the hom soln space has basis
\( B = \{ y_1, y_2 \}  \subset C^{(2)}(I) \). Here we require 

\[  y_p'' + py_p' + qy_p = f(x) \text{ (monic). } \]

\[ y_p = v_1y_1 + v_2y_2 \implies \]

\[ y_p' = v_1'y_1 + v_1y_1' + v_2'y_2 + v_2y_2' \]

If \( v_1'y_1 + v_2'y_2 = 0 \) then 

\[ y_p' = v_1y_1' + v_2y_2' \]

Thus, 

\[ y_p'' = v_1'y_1' + v_1y_1'' + v_2'y_2' v_2y_2'' \]

\[ y_p'' +py_p' + qy_p = f(x) \iff \]

\[ v_1'y_1' + v_1y_1'' + v_2'y_2' v_2y_2'' + pv_1y_1' + pv_2y_2' +
qv_2y_1 + qv_2y_2 = f(x) \iff \]

\[ v_1(y_1'' + py_1' + qy_1) + v_2(y_2'' + py_2' + qy_2) + v_1' y_1' +
v_2'y_2' = f(x) \iff \]

\[ v_1'y_1' + v_2'y_2' = f(x). \]

therefore

\[
\begin{cases}
  y_1' v_1' + y_1'v_2' = f(x) \\
  y_1v_1' + y_2v_2' = 0
\end{cases}
\]

iff 

\[ 
\begin{pmatrix}
  y_1 & y_2 \\
  y_1'& y_2' 
\end{pmatrix}
\begin{pmatrix}
  v_1'\\
  v_2' 
\end{pmatrix}
=
\begin{pmatrix}
  a\\
  b
\end{pmatrix}
\]

% missed some stuff

Therefore 

\[ v_1 = - \int \frac{y_2(x)f(x)}{|w(x)|} dx \]

and 

\[  v_2 =  \int \frac{y_1(x)f(x)}{|w(x)|} dx \]

Hence, 

\[ y_p(x) =  \]

% missed some more stuff

Now, in general, consider 

\[ \sum_{k=0}^{n} a_ky^{(k)} = f(x) \]
with basis \( B = \{ y_j \in C^{(n)} (I) | 1 \leq j \leq n \} \) for the
assoc hom ODE. Also, consider 

\[ y_p= \sum_{j=1}^{n} v_jy_j, \]
then
\begin{align*}
  y_p' &= \sum_{j=1}^{n}(v_j'y_j +v_jy_j') \\
  &= \sum_{j=1}^{n} v_jy_j' \text{ , if } \sum_{j=1}^{n}v_j'y_j = 0.
\end{align*}

Now, 

\begin{align*}
  y_p'' &= \sum_{j=1}^{n}(v_j'y_j' + v_jy_j'') \\
  &= \sum_{j=1}^{n}y_jy_j'' \text{ , if } \sum_{j=1}^{n}v_j'y_j' =0
\end{align*}

continuing, 

\[ y_p^{(k)} = \sum_{j=1}^{n} v_jy_j^{(k)} \text{ , if }
\sum_{j=1}^{n}v_j'y_j^{(k-1)} =0 \]

for \( 1 \leq k \leq n-1 \). Finally, 

\begin{align*}
  y_p^{(n)} &= \sum_{j=1}^{n} (v_j'y_j^{(n-1)} +v_jy_j^{(n)}) \\
 &=    \sum_{j=1}^{n} v_j'y_j^{(n-1)} + \sum_{j=1}^{n}v_jy_j^{(n)}) 
\end{align*}

Therefore

\[ \sum_{k=1}^{n}a_ky_p^{(k)} = f(x) \iff \]

\[ a_n \sum_{j=1}^{n} v_j'y_j^{(n-1)} + a_n  \sum_{j=1}^{n}v_jy_j^{(n)}
+ \sum_{k=1}^{n-1}a_k \sum_{j=1}^{n}v_jy_j^{(k)} = f(x) \iff\]

\[  a_n \sum_{j=1}^{n} v_j'y_j^{(n-1)} +  \sum_{j=1}^{n} a_nv_jy_j^{(n)}
+ \sum_{j=1}^{n} \sum_{k=1}^{n=1} a_kv_jy_j^{(k)} = f(x) \iff
 \]

 \[ 
  a_n \sum_{j=1}^{n} v_j'y_j^{(n-1)}  + \sum_{j=1}^{n}v_j \sum_{k=1}^{n}
  a_ky_j^{(k)} = f(x) \iff
 \]

 \[ a_n \sum_{j=1}^{n} v_n'y_j^{(n-1)} = f(x) \]

 since \( y_j \in B. \) Thus, 

 \begin{align*}
   \sum_{j=1}^{n}v_j'y_j &= 0  \text{ (1st eqn) }\\
   \sum_{j=1}^{n}v_j'y_j' &= 0 \text{ (2nd eqn) }\\
   \vdots \\
   \sum_{j=1}^{n}v_j'y_j^{(k-1)} &= 0 \text{ (kth eqn) }\\
   \vdots \\
   \sum_{j=1}^{n}v_j'y_j^{(n-2)} &= 0 \text{ ((n-1)th eqn) }\\
   a_n\sum_{j=1}^{n}v_j'y_j^{(n-1)} &= f(x) \text{ (nth eqn) }\\
 \end{align*}

 \[ \implies W(y_j)_{1 \leq j \leq n} 
 \begin{pmatrix}
   v_1'\\
   \vdots \\
   v_n'
 \end{pmatrix}
 =
 f(x) \vec{e_n}
 \]

 Notice that \( |W(y_j)_{1 \leq j \leq n}| \neq 0 \) for all \( x \in I
 \) since \( y_js \) are linearly independent  ( being that \( y_j \in B
 \)). Thus, by Cramer's Rule, 

 \[ v_j' = \frac{D_j}{D},  \]

 where \( D = |W(y_j)_{1 \leq j \leq n}| \) and

 \[ D_j =
 \begin{vmatrix}
   y_1 & \cdots & 0 & \cdots & y_n \\
   y_1' & \cdots & 0 & \cdots & y_n' \\
   y_1'' & \cdots & 0 & \cdots & y_n'' \\
   \vdots &       & \vdots & & \vdots \\
   y_1^{(n-1)} & \cdots & f(x) & \cdots & y_n^{(n-1)}
 \end{vmatrix}
 = (-1)^{n+j} f(x) |W_{n-1} (y_k)_{\substack{1 \leq k \leq n \\ j \neq
 k}}| 
 \]

 Therefore 

 \[ v_k' = (-1)^{j+n}f(x) \frac{|W_{n-1}(y_k)_{\substack{1 \leq k \leq n
 \\
 j \neq k}}|}{a_n|W_n(y_k)_{1 \leq k \leq n}|}  \]

 Thus, 

 \[ \text{ (1) } v_k(x) = 
  \frac{(-1)^{j+n}}{a_n} \int f(x) \frac{|W_{n-1}(y_k)_{\substack{1 \leq
  k \leq n \\
  j \neq k}}|}{a_n|W_n(y_k)_{1 \leq k \leq n}|}  dx,
 \]

Where \( y_p(x) = \sum_{k=1}^{n}v_k(x)y_k(x) \) is a particular soln to
\( \sum_{k=1}^{n}a_ky^{(k)} = f(x) \). (1) is called the variation of
parameters formulas. \\[5mm]

Ch 4 systems of ODEs 

\begin{example}
  Consider a curve \( \vec{r}:[a,b] \to \mathbb{R}^n \), where \(
  \vec{r}(t) = (x_1(t), \dots , x_n(t)) \) and \( t \in [a,b] \). By
  Newton's 2nd Law, \( \vec{F} = m\vec{a} = m\vec{r}''(t)\). Thus, 

  \[ 
  \begin{pmatrix}
    m(t)x_1''(t) \\
    m(t)x_2''(t) \\
    \vdots \\
    m(t)x_n''(t) 
  \end{pmatrix}
  =
  \begin{pmatrix}
    F_1(t) \\
    F_2(t) \\
    \vdots \\
    F_n(t) 
  \end{pmatrix}
  ,
  \]
  
  which is a system of \( n \) many 2nd order ODEs in \( (x_1, \dots ,
  x_n) \)


\end{example}



% }}}

\marginpar{Date: 04.21.15 } % Continue notes 

\marginpar{Date: 04.22.15 } % Continue notes 

\begin{example}
  Consider an 

  \[ \text{ (1) } y^{(n)}+ \sum_{k=0}^{n-1} p_k(x) = f(x), \]

  \[ y^{(n)} + p_{n-1}y^{(n-1)} + \cdots + p_2(x)y'' + p_1(x)y' +
  p_0(x)y = f(x). \]

  put \(y_k = y^{(k-1)}  \) for \( 1 \leq k \leq n+1 \), then 

  \begin{align*}
    y_1 &= y \implies \\
    y_2 &= y' = y_1' \implies \\
    y_3 &= y'' = y_1'' = y_2' \implies \\
    y_4 &= y''' = y_1''' = y_2'' = y_3'  \implies \\
    &\vdots \\
    y_{n+1} &= y^{(n)} = y_n'
  \end{align*}

  Thus, (1) becomes 

  \[ \text{ (2) } y_n' + \sum_{k=0}^{n-1}p_k(x)y_k' = f(x), \]

  which is a first order linear ODE. puting (2) whith the alone n-1
  many 1st order ODEs yields the following 1st order system: with n
  many ODEs:

  \[
  \begin{cases}
  y_n' + \sum_{k=0}^{n-1}p(x)y_k' = f(x)  \\
  y_1' = y_2 \\
  y_2' = y_3 \\
  \vdots \\
  y_{n-1}' = y_n
  \end{cases}
\]

Again, this is a system of n many 1st linear ODEs from a single nth
order linear ODE. This gives yet another example to motivate studying
systems of linear ODEs.

\end{example}

\begin{example}
  page 255 number 12 \\ 
  
 \[ 
  \begin{cases}
   x' = y  \\
  y' = x 
  \end{cases}
  \]
  
  Note: \( x = y' = x'' \). So, this yields \( x'' - x = 0 \), which has
  basis \( B = \{ \cosh t, \sinh t \} \). Thus, 

  \[ x = a \cosh t + b \sinh t \]

  Now, 

  \[ y =  b \cosh t + a \sinh t\]

  Notice that if \( \vec{r}(t) = (x(t), y(t))  \) then \( \vec{r}'(t) =
  (x'(t), y'(t)) = (y(t), x(t))  \). We want such an \( \vec{r}(t)  \). In otherwords,
  we may view the above system as a single. 1st order ODE of a
  parametric function. 

  Recall that 

  \[ \cosh (\alpha + t) = \cosh \alpha \cosh t + \sinh \alpha \sinh t \]

  Take \( A \in \mathbb{R} \) st \( a/A \geq 1 \), then \( a/A \in \)
  Range \( (\cosh)\). Now, we want \( \alpha \in \mathbb{R} \) st

  \[ a = A \cosh \alpha \text{ \& }  b = A \sinh \alpha.\]

  So, 

  \( \coth \alpha = \frac{ \cosh \alpha}{ \sinh \alpha} = \frac{a}{b} \)
  holds for some \( \alpha \) st \( |\alpha| < 1 \) since \( \text{
  Range} (\coth) = \mathbb{R} \). therefore for this \( \alpha \), 

  \begin{align*}
   x &=  a \cosh t + b \sinh t \\
   &= A \cosh \alpha \cosh t + A \sinh \alpha \sinh t \\
   &= A \cosh (\alpha + t).
  \end{align*}

  therefore \( x = A \cosh (\alpha + t) \); whence, \( y = x' = A \sinh
  (\alpha + t).\)

  Hence, 

  \[ x^2 - y^2 = A^2 \implies \frac{x^2}{A^2} - \frac{y^2}{A^2} = 1 \]

These are the solns to \( \vec{r}'(t) = (x'(t), y'(t)).  \) Note : \(
\text{ Range} (\cosh) = [1,\enfinity)\). So, if \( A > 0 \) then \( x(t)
> 0\); where, \( \vec{r}(t) \) coresponds to the righthand branch.
Whereas, if \( A<0 \) then \( \vec{r}(t) \) coresponds to the left hand
branch. 

Notice that we soved this linear system by "substitution." 

\end{example}

\marginpar{Date: 04.23.15 } % Continue notes 

\begin{example}
  (Recall). 

  let \( \vec{r}(t) = (x(t), y(t))\), and consider \( \vec{r}'(t) =
  (x'(t), y'(t)) \), then 

  \[
  \begin{cases}
    x'(t) = y(t) \\
    y'(t) = x(t)
  \end{cases}
  \iff
  \begin{cases}
    x'(t) - y(t) =0 \\
    y'(t) - x(t) =0
  \end{cases}
  \iff
  \begin{cases}
    L_{11}x + L_{12}y = 0 \\
    L_{21}x + L_{22}y = 0
  \end{cases}
  \iff
  \]

  \[
  \begin{pmatrix}
    L_{11} & L_{12} \\
    L_{21} & L_{22}
  \end{pmatrix}
  \begin{pmatrix}
    x \\
    y
  \end{pmatrix}
  = 
  \vec{0}
  \]

  Where \( L_{11} = D = L_{22} \) and \( L_{12} = -1 = L_{21} \) 

  In other symbols, 

  \[ 
  \begin{pmatrix}
    D & -1 \\
    -1 & D
  \end{pmatrix}
  \begin{pmatrix}
    x \\
    y
  \end{pmatrix}
  = 
  \vec{0}
  \]

  Recall that \( \mathbb{R}[D] \equiv \mathbb{R}[x]; \) in particular,
  \( \mathbb{R}[D] \) is communative "ring" (multiplicative
  communative).

  Solve by "elimination:"

  \[ L_{21}(L_{11}x + L_{12}y) = L_{21}0 \implies \]

  \[ L_{21}L_{11}x + L_{21}L_{12}y = L_{21}0 \]

  \[ L_{11}(L_{21}x + L_{21}y) = L_{11}0 \implies  \]

  \[ L_{11}L_{21}x + L_{11}L_{21}y = 0 \]

  \hline

  \[ L_{21}L_{12}y - L_{11}L_{22}y = 0 \implies \]

  \[ (L_{21}L_{12} - L_{11}L_{22}) y= 0 \implies \]

  \[ 
  \begin{vmatrix}
    L_{11} & L_{12} \\
    L_{21} & L_{22}
  \end{vmatrix}
  y
  = 0
  \]

  Anologosly 

  
  \[ 
  \begin{vmatrix}
    L_{11} & L_{12} \\
    L_{21} & L_{22}
  \end{vmatrix}
  x
  = 0
  \]

  Therefore

  \[ ( |L|I) 
  \begin{pmatrix}
    x \\
    y
  \end{pmatrix}
   = \vec{0}
  \]

  where

  \[
  L=
  \begin{pmatrix}
    L_{11} & L_{12} \\
    L_{21} & L_{22}
  \end{pmatrix}
  \]

  Recall: If A is a sq matrix then 

  \[ AA^a = |A|I \]

  This holds for any A over a commutative ring, e.g., 

  \[ A \in \text{ Mat }_n(R) \]

  where R is a commutative ring say \( R = \mathbb{C} \)

  Thus if \( L \in \text{ Mat}_n( \mathbb{R}[D]) \) then 

  \[ L^aL = |L|I_n, \]

  where \( I_n \text{ Mat}_n( \mathbb{R}[D]),  \)

\[ 
I_n = 
\begin{pmatrix}
  1 & \cdots & 0 \\
  \vdots &  & \vdots \\
  0 & \cdots & 1
\end{pmatrix}
\text{ , \& }
1_{Id} \in \mathbb{R}[D]
\]

consider \( L \in \text{ Mat}_n( \mathbb{R}[D]), \) \\

\( \vec{x}(t) = (x_1(t), \dots , x_n(t)) \), \( \vec{F}(t) = (F_1(t),
\dots, F_n(t)) \in (C^{(d)}(I))^n,\) where \( d= \text{ max} \{ \text{
deg }(L_{ij}) \text{ | } 1 \leq i \text{ , } j \leq n \} \) and \( L =
(L_{ij}),  \) and the system of linear ODEs

\[ \text{ (1) }
L \vec{x} = \vec{F} \iff 
\begin{pmatrix}
  L_{11} & L_{12} & \cdots L_{1n} \\
  L_{21} & L_{22} & \cdots L_{2n} \\
  \vdots & \\
  L_{n1} & L_{n2} & \cdots L_{nn} 
\end{pmatrix}
\begin{pmatrix}
  x_1(t) \\
  x_2(t) \\
  \vdots \\
  x_n(t)
\end{pmatrix}
=
\begin{pmatrix}
  F_1(t) \\
  F_2(t) \\
  \vdots \\
  F_n(t)
\end{pmatrix}
\]

Here 

\[ L_{ij} = \sum_{k=0}^{m_{ij}} a_{i_k j_k} D^k \in \mathbb{R}[D] \]

Applying the adjoint formula to (1) 

\[ \text{ (2) } (|L|I_n) \vec{x} = L^a \vec{F}. \]

Recall: \( A^a = (c_{ij})^T \), \( C_{ij} = (-1)^{i+j} |M_{ij}| \)

In component form, (2) says that 

\[ \text{ (3) } |L|x_i(t) = _i(L^a) \vec{F}(t) ,\]

where \( _i(L^a) \) is the ith row of \( L^a \). Notice that (3) is a
linear ODE in the single unknown funct \( x_i(t) \). therefore previous
methods can be used to solve (3). 
\end{example}

\marginpar{Date: 04.27.15 } % Continue notes 

Now, we consider first order linear systems, at first, without constant
coefficients. These have the form 

\[ \vec{x}'(t) + p(t) \vec{x}(t) = \vec{q}(t),  \]

where \( \vec{x}: I \subseteq \mathbb{R} \to \mathbb{R}^n \) and \( p(t)
\in \text{ Mat}_n (C(I)) \) and \( \vec{q} \in ( C^{(1)}(I))^n \). 

In "normal form" this becomes 

\[   \vec{x} \;'(t) = p(t) \vec{x}(t) + \vec{q}(t) \]



Aside: \( R[D] \) is a communative "ring." however, \( (C(t))[D] \) this
is not a communative, the point is that to use the det proputies
developed last semester, we need \( \text{ Mat}_n( \mathbb{R}) \), where
\( \mathbb{R} \) is commutative ring (then thos proff genral

\begin{example}
(\( ((C)(I))[D] \) is not commutative). 

Put

\[ L_1 = -tD+1 \]
\[ L_2 = t^2D - 3, \]

then \( L_1, L_2 \in ( C^{(\infty)}( \mathbb{R}) ) [D] \). Notice that 

\begin{align*}
  L_2x &= (t^2D-3)x \\
  &= t^2Dx-3x \\
  &= t^2x' -3x \implies
\end{align*}

\begin{align*}
  L_1L_2x &= (-tD + 1)L_2x \\
  &= (-tD + 1)(t^2Dx-3x) \\
  &= -tD(t^2Dx-3x) + t^2Dx-3x \\
  &= -tD(t^2Dx) + 3tDx + t^2Dx-3x \\
  &= -t(2tDx + t^2D^2x) + 3+ Dx + t^2Dx-3x \\
  &= -2t^2Dx - t^3D^2x + 3tDx + t^2Dx - 3x \\
  &= -t^2Dx - t^3D^2x + 3tDx - 3x. 
\end{align*}

Also, 

\begin{align*}
  L_1x &= (-tD+1)x \\
  &= -tDx + x \implies \\
  L_2L_1x &= (t^2D-3)L_1x \\
  &= t^2DL_1x - 3L_1x \\
  &= t^2D(-tDx+x) + 3tDx - 3x \\
  &= t^2D(-tDx) + t^2Dx + 3tDx-3x \\
  &= t^2( -Dx- tD^2x) + t^2Dx+3tDx - 3x \\
  &= -t^2Dx - t^2D^2x + t^2dx + 3tDx - 3x \\
  &= -t^3D^2x+ 3tDx-3x. 
\end{align*}

therefore \( L_1L_2 \neq L_2L_1. \)

\end{example}

\marginpar{Date: 04.29.15 } % Continue notes 

\[   \vec{x} \;'(t) = p(t) \vec{x}(t) + \vec{q}(t) \]

where \( p(t) \in \text{ Mat}_n (C(I)) \). Now, we consider the case
where \(  p(t) \in \text{ Mat}_n ( \mathbb{R})  \), i.e., the entries of
the matrix p are costant functions. In the hom case, this becomes 

\[ \vec{x} \;'(t) = A \vec{x}(t) \]

or

\[ \text{ (1) } \vec{x} \;' = A \vec{x} \]

where \( A \in \text{ Mat}_n ( \mathbb{R})  \). \\

Notice that if \( n = 1 \) then (1) becomes 

\[ x' = ax, \]

Which has solns  \( x(t) = ke^{at}. \) This leads us to a conjecture in
the more general case, 

\[ x_i(t) = k_ie^{a_it} \text{ for } 1 \leq i \leq n.\]

In other words, 

\[ \vec{x}(t) =
\begin{pmatrix}
  k_1e^{a_1t} \\
  k_2e^{a_2t} \\
  \vdots \\
  k_ne^{a_nt} 
\end{pmatrix}\]

Since \( \vec{x}' = (x_i') \), 

\[ \vec{x}'(t) =
\begin{pmatrix}
  a_1k_1e^{a_1t} \\
  a_2k_2e^{a_2t} \\
  \vdots \\
  a_nk_ne^{a_nt} 
\end{pmatrix}\]

therefore this \( \vec{x}(t) \) is a soln to (1) \iff

\[ 
\begin{pmatrix}
  a_1k_1e^{a_1t} \\
  a_2k_2e^{a_2t} \\
  \vdots \\
  a_nk_ne^{a_nt} 
\end{pmatrix}
=A
\begin{pmatrix}
  k_1e^{a_1t} \\
  k_2e^{a_2t} \\
  \vdots \\
  k_ne^{a_nt} 
\end{pmatrix}
\]

Here we see that if \( a_i = a_j \) for all i and j , say \( a = a_i\), then
\( \vec{x}'(t) = a\vec{x}(t) \) and so, (1) becomes 

\[ a\vec{x}(t) = A\vec{x}(t) \]

Now, notice that 

\[ \vec{x}(t) = e^{at}\vec{k}, \]

where \( \vec{k} = (k_1, k_2, \dots , k_n) \). therefore (1) now becomes

\[ ae^{at}\vec{k} = A(e^{at}\vec{k}) \implies  \]

\[ A\vec{k} = a\vec{k}. \]

Recall: an eigenvalue of \( A \in \text{ Mat}_n( \mathbb{R}) \) is
\( \lambda \in \mathbb{R} \) iff there is \( \vec{0} \neq \vec{v} \in
\mathbb{R}^n\) st 

\[ A\vec{v} = \lambda \vec{v}. \]

Here \( \vec{v} \) is called an eigevector assoc with \( \lambda. \)
therefore from above, we see that \( \vec{x}(t) = e^{\lambda t}\vec{v}
\) is a soln to \( \vec{x}' = A \vec{x} \) iff \( \lambda \) is an
eiginvalue of \( A \) and \( \vec{v} \) is an assoc eigenvector of \(
\lambda. \) \\

Recall: to find eigenvalues of \( A \in \text{ Mat}_n( \mathbb{R}) \),
notice that for \( \vec{v} \neq \vec{0}, \)

\begin{align*}
  A\vec{v} \lambda \vec{v} &\iff \lambda \vec{v} - A \vec{v} = \vec{0}\\
  &\iff (\lambda I_n  - A) \vec{v} = \vec{0}, 
\end{align*}

which is a hom system with a nontivial soln \( \vec{v} \). So, therefore

\[ \text{ det}(\lambda I_n -A) = 0.\]

Also, recall that the char poly of \(  A \in \text{ Mat}_n( \mathbb{R}),  \)

\[ p_A(x) = \text{ det}(xI_n-A) \in \mathbb{R}[x], \]

which is an nth degree poly over \( \mathbb{R} \). So, therefore to find
eigenvalues of \( A \), we must find the zeros of \( p_A(x) \). Once we
have the eigenvalue \( \lambda \), to find an assoc \( \vec{v} \pm
\vec{0} \), we solve for \( \vec{v} \) in \( (A - \lambda I_n) \vec{v} =
 \vec{0}\). 

